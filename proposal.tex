\documentclass{article} % For LaTeX2e
\usepackage{nips15submit_e,times}
\usepackage{hyperref}
\usepackage{url}
\usepackage[inline]{enumitem}
%\documentstyle[nips14submit_09,times,art10]{article} % For LaTeX 2.09


\title{A Survey on Differential Privacy}


\author{
Achal Dave* \\
\texttt{achald@cs.cmu.edu} \\
\And
Jingyan Wang* \\
\texttt{jingyanw@cs.cmu.edu} \\
\\
}

% The \author macro works with any number of authors. There are two commands
% used to separate the names and addresses of multiple authors: \And and \AND.
%
% Using \And between authors leaves it to \LaTeX{} to determine where to break
% the lines. Using \AND forces a linebreak at that point. So, if \LaTeX{}
% puts 3 of 4 authors names on the first line, and the last on the second
% line, try using \AND instead of \And before the third author name.

\newcommand{\fix}{\marginpar{FIX}}
\newcommand{\new}{\marginpar{NEW}}

\nipsfinalcopy % Uncomment for camera-ready version

\begin{document}

\maketitle

\section{Introduction}

The growth of the internet and electronic databases in general has led to an
increasing concern about the privacy of users' information. These databases
enable researchers and corporations alike to learn information, such as
identifying new trends or learning to recommend movies, from large
groups of people. Although increasingly useful, these methods come at the cost
of users' privacy: the more questions we ask about a large dataset of users'
information, the more information we receive about each individual user's
information. Prior work, including \cite{narayanan2008robust,
sweeney1997weaving, ganta2008composition}, has shown that even anonymous
datasets can reveal sensitive details about individuals. Differential privacy
aims to provide provable privacy for users with high probability, while allowing
queries that reveal details about the dataset as a whole.

\section{Reading List}

We plan on reading and reviewing the following
\begin{enumerate}
\item One example of the need for differential privacy. \cite{narayanan2008robust}
\item A survey of differential privacy results. \cite{dwork2008differential}
\item An application of differential privacy in empirical risk minimization.
\cite{chaudhuri2011differentially}
\item Excerpts from a book on the fundamentals of differential privacy
\cite{dwork2014algorithmic}
\end{enumerate}

{\small
\bibliographystyle{plain}
\bibliography{proposal}
}
\end{document}
